\section{Ziel}
Ziel des Versuches war es die Zusammensetzung eines 3x3-Würfels in einer Ebene zu bestimmen, wobei die einzelnen Teilwürfel aus unterschiedlichen Metallen besteht. 

\section{Theorie}
\subsection{Tomographie}
Die Tomographie ist ein Bild-gebendes Verfahren, welches viel Anwendung in der heutigen Medizin findet. Besonders die so genannte Computertomographie, kurz CT, ist weit 
verbreitet.
Durch dieses Verfahren werden Querschnitte erzeugt und durch die Untersuchung mehrerer Schichten kann so ein 3 Dimesnionales Bild generiert werden.

\noindent
Im Allgemeinen wird für die Tomographie $\gamma$-Strahlung benutzt. Durch die unterschiedlichen Absorptionskoeffizienten und durch die Bestrahlung des Targets aus 
verschiedenen Winkeln kann ein Bild erzeugt werden.

\subsection{Wechselwirkung von Materie mit Gamma-Strahlung}
$\gamma$-Strahlung wechselt wirkt hauptsächlich in 3 Art und Weisen mit Materie. Diese sind der Photoeffekt, die Compton-Streuung, sowie die Paarerzeugung.
Beim Photoeffekt wird ein Photon vollständig von einem gebundenen Elektron absorbiert, sodass dieses aus seiner Bindung herausgelöst wird. 
daher dominiert im Allgemeinen bei einer Energie <100keV 

\subsection{Fehlerbestimmung}